%% If you need to pass whatever options to xcolor
\PassOptionsToPackage{dvipsnames}{xcolor}

%% If you are using \orcid or academicons
%% icons, make sure you have the academicons 
%% option here, and compile with XeLaTeX
%% or LuaLaTeX.
% \documentclass[10pt,a4paper,academicons]{altacv}

%% Use the "normalphoto" option if you want a normal photo instead of cropped to a circle
% \documentclass[10pt,a4paper,normalphoto]{altacv}

\documentclass[10pt,a4paper]{altacv}
%% AltaCV uses the fontawesome and academicon fonts
%% and packages. 
%% See texdoc.net/pkg/fontawecome and http://texdoc.net/pkg/academicons for full list of symbols.
%% 
%% Compile with LuaLaTeX for best results. If you
%% want to use XeLaTeX, you may need to install
%% Academicons.ttf in your operating system's font 
%% folder.

% Change the page layout if you need to
\geometry{left=1cm,right=9cm,marginparwidth=6.8cm,marginparsep=1.2cm,top=1.25cm,bottom=1.25cm,footskip=2\baselineskip}

% Change the font if you want to.

% If using pdflatex:
\usepackage[T1]{fontenc}
\usepackage[utf8]{inputenc}
\usepackage[default]{lato}

% If using xelatex or lualatex:
% \setmainfont{Lato}

% Change the colours if you want to
\definecolor{Mulberry}{HTML}{72243D}
\definecolor{SlateGrey}{HTML}{2E2E2E}
\definecolor{LightGrey}{HTML}{666666}
\colorlet{heading}{Sepia}
\colorlet{accent}{Mulberry}
\colorlet{emphasis}{SlateGrey}
\colorlet{body}{LightGrey}

% Change the bullets for itemize and rating marker
% for \cvskill if you want to
\renewcommand{\itemmarker}{{\small\textbullet}}
\renewcommand{\ratingmarker}{\faCircle}

\usepackage[colorlinks]{hyperref}

\begin{document}

\name{Julius Colliander Celik}
\tagline{Master student in Software Engineering of Distributed Systems}
\photo{2.8cm}{profile-square}
\personalinfo{%
  % Not all of these are required!
  % You can add your own with \printinfo{symbol}{detail}
  \email{julius.cc@hotmail.com}
  \phone{+46 762 35 35 07}
  \mailaddress{Borgarfjordsgatan 21 A lgh 4208, 164 53 Kista}
  \location{Stockholm, SWEDEN}
%   \homepage{www.homepage.com}
%   \twitter{@twitterhandle}
  \linkedin{linkedin.com/in/jcelik}
  \github{github.com/juliuscc}
  %% You MUST add the academicons option to \documentclass, then compile with LuaLaTeX or XeLaTeX, if you want to use \orcid or other academicons commands.
%   \orcid{orcid.org/0000-0000-0000-0000}
}

%% Make the header extend all the way to the right, if you want. 
\begin{fullwidth}
\makecvheader
\end{fullwidth}

%% Depending on your tastes, you may want to make fonts of itemize environments slightly smaller
\AtBeginEnvironment{itemize}{\small}


%% Provide the file name containing the sidebar contents as an optional parameter to \cvsection.
%% You can always just use \marginpar{...} if you do
%% not need to align the top of the contents to any
%% \cvsection title in the "main" bar.
\cvsection[page1sidebar]{Experience}

\cvevent{Programmer}{LS Elektronik AB}{October 2016 -- August 2018}{Spånga, Stockholm}
\textit{LS Elektronik AB is an engineering and manufacturing company. They have a broad range of products with a focus on radio technology and radio over IP.}
\begin{itemize}
    \item I developed a web interface used for configuring network settings on a hardware product.
    \item I analyzed and optimized one of the primary products, the Mimer SoftRadio desktop client, with over a 100x improvement, enabling the program to handle vastly more radio interfaces.
\end{itemize}

\divider

\cvevent{Bachelor Thesis}{LS Elektronik AB}{March 2018 -- June 2018}{Spånga, Stockholm}
The thesis was conducted as a case study that examined a large set of existing tools that use JSON Schemas to generate graphical user interfaces.

\cvsection{Education}

\cvevent{Degree Program Information and Communication Technology}{Royal Institute of Technology (KTH)}{2015 -- 2020}{Stockholm}
\begin{itemize}
    \item BSc Information and Communication Technology
    \item MSc Software Engineering of Distributed Systems
\end{itemize}

\divider

\cvevent{Information and media technology in secondary school}{Stockholms tekniska gymnasium}{2012 -- 2015}{}

\cvsection{Volunteer experience}

\cvevent{Team Leader - IT and Logistics}{THS Mental Health Week}{2019}{Stockholm}

\divider

\cvevent{Active in Student Union}{Chapter For Information- and Nanotechnology (KTH)}{2016, 2017}{Kista, Stockholm}
\begin{itemize}
    \item Planned reception activities for 150 new students.
    \item As mentor for an international student group, we achieved unrepresented levels of attendance during the student reception. It was the first year ever that the two student groups were not merged during the student reception, and our group had the highest attendance.
\end{itemize}

\cvsection[page2sidebar]{Projects}

\cvevent{\printinfo{\faSearch}{Simple Search Engine}}{}{}{}
A simple search engine using web crawler data from Common Crawl, to index websites and sort them. The project resulted in very successful results with an extremely small subset of the available data.

\vspace{6pt}

\begin{itemize}
    \item \textbf{Technologies used:} Spark, Hadoop, Cassandra and Scala.
    \item \textbf{Project link:} \href{https://github.com/hannesrabo/simple-search-engine}{https://github.com/hannesrabo/simple-search-engine}
\end{itemize}

\divider

\cvevent{\printinfo{\faVolumeUp}{heos-api}}{}{}{}
The leading npm package to control your Denon Heos smart speakers, with over 500 downloads a month. It is a library that enables developers to write applications in Node.js that communicates with and controls Denon Heos smart speakers.

\vspace{6pt}

\begin{itemize}
    \item \textbf{Technologies used:} JavaScript, NPM, Node.js, Jest and Travis CI.
    \item \textbf{Project link:} \href{https://github.com/juliuscc/heos-api}{https://github.com/juliuscc/heos-api}
\end{itemize}

\divider


\cvevent{\printinfo{\faApple}{cputemp-macos}}{}{}{}
A MacOS command line tool to read or log the cpu temperature.

\vspace{6pt}

\begin{itemize}
    \item \textbf{Technologies used:} JavaScript, NPM, Node.js and Commander.
    \item \textbf{Project link:} \href{https://github.com/juliuscc/cputemp-macos}{https://github.com/juliuscc/cputemp-macos}
\end{itemize}


\end{document}
