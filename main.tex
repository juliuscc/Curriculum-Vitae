%% If you need to pass whatever options to xcolor
\PassOptionsToPackage{dvipsnames}{xcolor}

%% If you are using \orcid or academicons
%% icons, make sure you have the academicons 
%% option here, and compile with XeLaTeX
%% or LuaLaTeX.
% \documentclass[10pt,a4paper,academicons]{altacv}

%% Use the "normalphoto" option if you want a normal photo instead of cropped to a circle
% \documentclass[10pt,a4paper,normalphoto]{altacv}

\documentclass[10pt,a4paper]{altacv}
%% AltaCV uses the fontawesome and academicon fonts
%% and packages. 
%% See texdoc.net/pkg/fontawecome and http://texdoc.net/pkg/academicons for full list of symbols.
%% 
%% Compile with LuaLaTeX for best results. If you
%% want to use XeLaTeX, you may need to install
%% Academicons.ttf in your operating system's font 
%% folder.


% Change the page layout if you need to
\geometry{left=1cm,right=9cm,marginparwidth=6.8cm,marginparsep=1.2cm,top=1.25cm,bottom=1.25cm,footskip=2\baselineskip}

% Change the font if you want to.

% If using pdflatex:
\usepackage[T1]{fontenc}
\usepackage[utf8]{inputenc}
\usepackage[default]{lato}

% If using xelatex or lualatex:
% \setmainfont{Lato}

% Change the colours if you want to
\definecolor{Mulberry}{HTML}{72243D}
\definecolor{SlateGrey}{HTML}{2E2E2E}
\definecolor{LightGrey}{HTML}{666666}
\colorlet{heading}{Sepia}
\colorlet{accent}{Mulberry}
\colorlet{emphasis}{SlateGrey}
\colorlet{body}{LightGrey}

% Change the bullets for itemize and rating marker
% for \cvskill if you want to
\renewcommand{\itemmarker}{{\small\textbullet}}
\renewcommand{\ratingmarker}{\faCircle}

\usepackage[colorlinks]{hyperref}

\begin{document}

\name{Julius Colliander Celik}
\tagline{Master student in Software Engineering of Distributed Systems}
\photo{2.8cm}{profile-square}
\personalinfo{%
  % Not all of these are required!
  % You can add your own with \printinfo{symbol}{detail}
  \email{jcelik@kth.se}
  \phone{+46 762 35 35 07}
  \mailaddress{Borgarfjordsgatan 21 A lgh 4208, 164 53 Kista}
  \location{Stockholm, SWEDEN}
%   \homepage{www.homepage.com}
%   \twitter{@twitterhandle}
  \linkedin{linkedin.com/in/jcelik}
  \github{github.com/juliuscc}
  %% You MUST add the academicons option to \documentclass, then compile with LuaLaTeX or XeLaTeX, if you want to use \orcid or other academicons commands.
%   \orcid{orcid.org/0000-0000-0000-0000}
}

%% Make the header extend all the way to the right, if you want. 
\begin{fullwidth}
\makecvheader
\end{fullwidth}

%% Depending on your tastes, you may want to make fonts of itemize environments slightly smaller
% \AtBeginEnvironment{itemize}{\small}


%% Provide the file name containing the sidebar contents as an optional parameter to \cvsection.
%% You can always just use \marginpar{...} if you do
%% not need to align the top of the contents to any
%% \cvsection title in the "main" bar.
\cvsection[page1sidebar]{Experience}

\cvevent{Programmer}{LS Elektronik AB}{October 2016 -- August 2018}{Spånga, Stockholm}
\textit{LS Elektronik AB is an engineering and manufacturing company developing electronic products for the professional market all over the world. They have a broad range of products with a slight focus on radio technology and radio over IP.}
\begin{itemize}
    \item I developed a web interface used for configuring network settings on a hardware product.
    \item I analyzed and optimized one of the primary products, the Mimer SoftRadio desktop client, with over a 100x improvement, enabling the program to handle vastly more radio interfaces.
    \item I introduced functionality on a range of products in the Mimer SoftRadio ecosystem in the programming language Delphi.
\end{itemize}

\divider

\cvevent{Telemarketer}{MySafety}{June 2016 -- July 2016}{Gärdet, Stockholm}

\divider

\cvevent{Study Coach}{My Academy}{December 2015 -- June 2016}{Stockholm, Sweden}

\divider

\cvevent{Breakfast Host}{Belgo Bar & Restaurant AB}{October 2012 -- Mars 2014}{Stockholm, Sweden}


\cvsection{IT-Competence}

\cvevent{\printinfo{\faPaintBrush}{Front end web development}}{}{}{}
\begin{itemize}
    \item JavaScript
    \item React.js
    \item Elm
    \item Pug (previously called Jade) and HTML
    \item SCSS, Stylus, PostCss and CSS
\end{itemize}

\divider

\cvevent{\printinfo{\faCode}{Back end web development}}{}{}{}
\begin{itemize}
    \item Node.js
    \item PHP
    \item MongoDB
    \item Webpack, NPM and gulp.js
    \item Jest testing framework
\end{itemize}

\divider

\cvevent{\printinfo{\faTerminal}{Other programming skills}}{}{}{}
\begin{itemize}
    \item Test driven development
    \item Git and Apache Subversion (SVN)
    \item Continuous integration with tools like Travis CI
    \item Java and Scala
    \item Elixir / Erlang
    \item C
    \item Linux
\end{itemize}

\end{document}
