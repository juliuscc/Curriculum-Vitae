%% If you need to pass whatever options to xcolor
\PassOptionsToPackage{dvipsnames}{xcolor}

%% If you are using \orcid or academicons
%% icons, make sure you have the academicons 
%% option here, and compile with XeLaTeX
%% or LuaLaTeX.
% \documentclass[10pt,a4paper,academicons]{altacv}

%% Use the "normalphoto" option if you want a normal photo instead of cropped to a circle
% \documentclass[10pt,a4paper,normalphoto]{altacv}

\documentclass[10pt,a4paper]{altacv}
%% AltaCV uses the fontawesome and academicon fonts
%% and packages. 
%% See texdoc.net/pkg/fontawecome and http://texdoc.net/pkg/academicons for full list of symbols.
%% 
%% Compile with LuaLaTeX for best results. If you
%% want to use XeLaTeX, you may need to install
%% Academicons.ttf in your operating system's font 
%% folder.

% Change the page layout if you need to
\geometry{left=1cm,right=9cm,marginparwidth=6.8cm,marginparsep=1.2cm,top=1.25cm,bottom=1.25cm,footskip=2\baselineskip}

% Change the font if you want to.

% If using pdflatex:
\usepackage[T1]{fontenc}
\usepackage[utf8]{inputenc}
\usepackage[default]{lato}

% If using xelatex or lualatex:
% \setmainfont{Lato}

% Change the colours if you want to
\definecolor{Mulberry}{HTML}{72243D}
\definecolor{SlateGrey}{HTML}{2E2E2E}
\definecolor{LightGrey}{HTML}{666666}
\colorlet{heading}{Sepia}
\colorlet{accent}{Mulberry}
\colorlet{emphasis}{SlateGrey}
\colorlet{body}{LightGrey}

% Change the bullets for itemize and rating marker
% for \cvskill if you want to
\renewcommand{\itemmarker}{{\small\textbullet}}
\renewcommand{\ratingmarker}{\faCircle}

\usepackage[colorlinks]{hyperref}

\begin{document}

\name{Julius Colliander Celik}
\tagline{Fullstack Developer and Tech Lead}
\photo{2.8cm}{profile-square}
\personalinfo{%
  % Not all of these are required!
  % You can add your own with \printinfo{symbol}{detail}
  \email{julius.cc@hotmail.com}
  \phone{+46 762 35 35 07}
  \mailaddress{Borgarfjordsgatan 21 A lgh 4208, 164 53 Kista}
  \location{Stockholm, SWEDEN}
%   \homepage{www.homepage.com}
%   \twitter{@twitterhandle}
  \linkedin{linkedin.com/in/jcelik}
  \github{github.com/juliuscc}
  %% You MUST add the academicons option to \documentclass, then compile with LuaLaTeX or XeLaTeX, if you want to use \orcid or other academicons commands.
%   \orcid{orcid.org/0000-0000-0000-0000}
}

%% Make the header extend all the way to the right, if you want. 
\begin{fullwidth}
\makecvheader
\end{fullwidth}

%% Depending on your tastes, you may want to make fonts of itemize environments slightly smaller
\AtBeginEnvironment{itemize}{\small}


%% Provide the file name containing the sidebar contents as an optional parameter to \cvsection.
%% You can always just use \marginpar{...} if you do
%% not need to align the top of the contents to any
%% \cvsection title in the "main" bar.
\cvsection[page1sidebar]{Experience}

\cvevent{IT Consultant / Tech Lead}{Netlight}{October 2020 -- Present}{}

I work with building a e-commerce platform using Node, Typescript, Koa, and Vue. My main focus is evenly divided between being a tech lead for a 5-person team and being a part of a DevOps transition team. I form long term technical strategies with other tech leads, and identify skill gaps in our team and coach my team members in their development.

\divider

\cvevent{Master Thesis Student and Software Engineer}{Digital Route}{January 2020 -- August 2020}{}

Initially I conducted my master thesis which was about enabling developers to generate microfrontends that are version- and type-safe, as well as framework agnostic (more on the next page). After that I lead a project of modernizing an internal service.
\divider

\cvevent{Teaching Assistant}{Royal Institute of Technology (KTH)}{August 2019 -- January 2020}{}

\begin{itemize}
    \item ID2207 Modern Methods in Software Engineering
    \item ID2209 Distributed Artificial Intelligence and Intelligent Agents
\end{itemize}

I held tutorials, coached students, managed administrative tasks and performed grading. Additionally, I lectured about microservices. 

\divider

\cvevent{Summer Intern}{Digital Route}{Summer 2019}{}
DigitalRoute provides platforms to process usage based data. I worked with their web based product CE and worked with Node.js, Express, React, TypeScript and Kubernetes. Additionally I automated publishing of their npm libraries.

\divider


\cvevent{IT Consultant}{We Know IT}{February 2019 -- June 2019}{}
A student consultant company focusing on web and app development. In addition to working part time with client projects, I held internal training about React and Node.js.

\divider

\cvevent{Developer}{LS Elektronik AB}{October 2016 -- August 2018}{}
LS Elektronik AB is an engineering and manufacturing company, with a focus on radio technology and radio over IP. I worked part time while studying, and full time over two summers.


\cvsection[page2sidebar]{Projects}

\cvevent{\printinfo{\faSearch}{Simple Search Engine}}{}{}{}
A simple search engine using web crawler data from Common Crawl, to index websites and sort them. The project resulted in very successful results with an extremely small subset of the available data.

\vspace{6pt}

\begin{itemize}
    \item \textbf{Technologies used:} Spark, Hadoop, Cassandra and Scala.
    \item \textbf{Project link:} \href{https://github.com/hannesrabo/simple-search-engine}{https://github.com/hannesrabo/simple-search-engine}
\end{itemize}

\divider

\cvevent{\printinfo{\faVolumeUp}{heos-api}}{}{}{}
The leading npm package to control your Denon Heos smart speakers, with over 500 downloads a month. It is a library that enables developers to write applications in Node.js that communicates with and controls Denon Heos smart speakers.

\vspace{6pt}

\begin{itemize}
    \item \textbf{Technologies used:} JavaScript, NPM, Node.js, Jest and Travis CI.
    \item \textbf{Project link:} \href{https://github.com/juliuscc/heos-api}{https://github.com/juliuscc/heos-api}
\end{itemize}

\divider


\cvevent{\printinfo{\faApple}{cputemp-macos}}{}{}{}
A MacOS command line tool to read or log the cpu temperature.

\vspace{6pt}

\begin{itemize}
    \item \textbf{Technologies used:} JavaScript, NPM, Node.js and Commander.
    \item \textbf{Project link:} \href{https://github.com/juliuscc/cputemp-macos}{https://github.com/juliuscc/cputemp-macos}
\end{itemize}


\end{document}
